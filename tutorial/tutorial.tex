\documentclass[11pt]{article}

\usepackage{acronym}
\usepackage{verbatim}
\usepackage{geometry}
\usepackage{listings}

\newcommand{\pythode}{\texttt{pythODE}}
\newcommand{\pypp}{\texttt{pythODE++}}
\newcommand{\adolc}{\texttt{ADOL-C}}
\newcommand{\umfpack}{\texttt{UMFPACK}}
\newcommand{\colpack}{\texttt{ColPack}}
\newcommand{\gnuplot}{\texttt{gnuplot}}
\newcommand{\numpy}{\texttt{numpy}}
\newcommand{\scipy}{\texttt{scipy}}

\acrodef{IVP}{initial-value problem}
\acrodef{OO}{object-oriented}
\acrodef{PSE}{problem-solving environment}

\title{Getting Started with \pypp}
\author{Adam Preuss and Raymond J. Spiteri}

\begin{document}
\maketitle

\section{Introduction}
\label{sec:intro}

The \pypp\ \ac{PSE} is designed to evaluate permutations of numerical
methods and \acp{IVP}\footnote{Recall that an \ac{IVP} is comprised of
  an ordinary differential equation and an initial condition.}.  It is
a (mostly) stand-alone collection of scripts and programs that is
heavily based on the functionality of a PSE that is entirely written
in python, \pythode.  The \pypp\ PSE is designed to be
performance-focused with an emphasis on runtime measurements of
numerical experimentation, whereas \pythode\ is geared more toward
performing in-depth and highly customizable analysis.

The numerical methods and \acp{IVP} in \pypp\ are written entirely in
C++. The supporting execution and analysis scripts are written in
Python. A general overview of the software, motivation, and specific
examples of application is presented in~\cite{preuss14}.  This
tutorial aims to introduce a user to the basics of behind implementing
\acp{IVP} and numerical methods in \pypp.  This tutorial is not
complete documentation for each software component.  However, there
are many examples of \acp{IVP} and numerical methods already
implemented in \pypp\ that can be used as a starting point for
implementing new, more complicated problems and methods.

\subsection{Requirements}

\subsubsection{System}

The minimalistic version of \pypp\ requires a C++ compiler, Python
version 2.7.3 with \numpy\ and \scipy\ modules, and the ability to
link against the standard library. Therefore, \pypp\ should be
supported by any UNIX-like platform (e.g., Linux, OpenBSD), Windows
(via Cygwin), or Mac OS X. At present, the \pypp\ PSE is run
exclusively from the command line.

The \pypp PSE can optionally link libraries that are used to support
the solution of sparse linear systems and (sparse) automatic
differentiation.  It is necessary to have \adolc\ installed if
automatic differentiation is desired. To use sparse matrices,
\umfpack\ is required, along with its supporting library for sparsity,
\colpack. Analysis graphs are generated using \gnuplot; therefore,
\gnuplot\ should be installed. Specific instructions for installing
these packages is generally operating-system dependent. Some common
examples on how to install these optional libraries are given in
Section~\ref{installation}.

\subsubsection{Background Knowledge}

To have a sense of the function of the supporting code in this PSE,
the user should be relatively comfortable with general programming
concepts such as virtual memory, lists, hashes, and function pointers,
as well as \ac{OO} concepts including (abstract) classes, inheritance,
polymorphism, and operator overloading.  Further, an understanding the
basic concepts of efficient programming with respect to the machine
cache is highly beneficial. Ignoring caching effects can cause severe
loss of performance.

As discussed in Section~\ref{sec:intro}, the \pypp\ \ac{PSE} is
written in a combination of C++ and Python.  This software makes use
of many advanced C++ concepts.  Although the implementation of
numerical methods and \acp{IVP} is not generally complicated, much of
the supporting code uses such concepts.  Specific examples include
templated classes/functions and exception handling.

\subsection{Software Design}

The software is organized into the following directories:
\begin{itemize}
\item \verb=analysis= contains C++ code for loading problem runs from disk and performing analysis passes on them.
\item \verb=core= contains all supporting functions and classes for vectors, matrices, hashes, lists, file input/output, etc.
\item \verb=ivps= contains implementations for all \acp{IVP} in \pypp.
\item \verb=loaders= contains supporting code that maintains a registry of all solvers, methods, and IVPs. This directory also contains the entry point for all components of the software such as the runner and the analysis tools.
\item \verb=methods= contains implementations for all methods in \pypp.
\item \verb=runner= contains code for running a set of parameters.
\item \verb=scripts= contains Python scripts for each of the numerical experiments.
\item \verb=solvers= contains implementations for all solvers in \pypp.
\item \verb=tutorial= contains the \LaTeX\ source associated with this document.
\end{itemize}

\section{Installation}
\label{installation}

The software can presently be accessed via the Numerical Simulation
Lab's subversion server. On your local system, navigate to the
location in which \pypp\ is to be installed and execute the following
command:
\begin{verbatim}
svn co svn+ssh://<username>@simcity.usask.ca/Users/svn/pythODE/pythODE++
\end{verbatim}

\subsection{Compiling}

The \pypp\ PSE is compiled using the \verb=build= script located at
the top level of the \pypp\ repository. Simply invoke this script (by
typing \verb=./build=) to compile. Arguments to the builder script
include \verb=debug= to compile with special debug information and
\verb=clean= to completely remove all executables and intermediate
build files.

By default, the software assumes that it will be built linking
libraries for \adolc\ and \umfpack. If these libraries are not
installed, the build script should be called with arguments
\verb=noadolc= and/or \verb=nosparsity= .

\subsection{Optional Libraries}

Installation instructions for the optional libraries are presented for Ubuntu Linux and for Mac OSX. The \pypp\ PSE has been extensively tested on both of these operating systems. It is recommended that the latest version of these libraries be used; however, during the development of \pypp, no issues involving the library versions were observed.

\subsubsection{Ubuntu Linux}

Installation of \gnuplot\ and \umfpack\ is simple on Ubuntu Linux because these libraries exist as precompiled packages. They can be installed as follows (when run as root):
\begin{verbatim}
apt-get install gnuplot libsuitesparse-dev
\end{verbatim}

\subsubsection{Mac OSX}

\section{Scripting}

The scripts contain many examples that evaluate numerical methods on
IVPs. In general, performing an experiment consists of two
parts. First, the runner loops over a set of specified runs. Second,
the analysis modules are used to gather runs and generate
meaningful graphs.

Each numerical experiment can be contained in a single Python
module. A numerical experiment can be invoked by using the
\verb=run-experiment.sh= script and specifying the module name
(without the \verb=.py=) as the argument. This module defining the
numerical experiment must specify the following two global variables:
\begin{itemize}
\item \verb=simname= is the name of the numerical simulation. It is
  used in directory names and can generally be thought of as a unique
  identifier for a set of similar numerical experiments. This name
  should probably not contain spaces, only because file management is
  more fragile when file names contain spaces.
\item \verb=simpath= is the path for the numerical simulation. All
  files associated with the simulation are stored in this path. The
  auto-runner creates new directories within this path each time a
  numerical simulation is conducted; therefore, the simulation path
  can shared for all \pypp\ simulations. For clean file management, it
  might be useful (though not necessary) to add \verb=simname= as a
  subdirectory \verb=simpath=, as is done in virtually all of the
  examples in \pypp.
\end{itemize}

There are two important functions required to specify a numerical
simulation. The first construct a list of run parameters; the second
specifies the analysis passes.

Run parameters are specified by the function \verb=GenerateRunList()=,
which returns a list of hashes. Each hash specifies the set of
parameters for the run. What follows is a list of parameters that are
commonly used. All of these are not required; however, common sense
must be invoked for solving IVPs when deciding whether a given
parameter combination is valid, e.g., specifying a constant solver
with a predictive step controller is not valid. Additional method- or
problem-specific parameters may be specified as well. The following
example shows how to instruct \pypp\ to solve two IVPs using two
methods.
\begin{lstlisting}[tabsize=4,language=Python]
def GenerateRunList():
	runlist = []
	for ivp in ('Brusselator1D','Brusselator2D'):
		for method in ('RK4','DOPR54'):
			runlist.append({'ivp':ivp,
							'method':method,
							'solver':'ConstantSolver',
							'dt':1e-2})
	return runlist
\end{lstlisting}

\begin{center}
\begin{tabular}{|p{.25\linewidth}|p{.65\linewidth}|}
\hline
\multicolumn{2}{|c|}{Required Parameters} \\
\hline \hline
\verb=ivp= & The (registered) name of the IVP that is to be solved, e.g., \verb=Brusselator1D=.\\\hline
\verb=method= & The (registered) name of the method that is to be used, e.g., \verb=RK4=. \\\hline
\verb=solver= & The solver that is to be used. This value can be one of \verb=ConstantSolver=, \verb=StepDoublingSolver=, or \verb=EmbeddedSolver=. \\\hline
\hline
\multicolumn{2}{|c|}{(Generally) Optional Parameters} \\
\hline \hline
\verb=dt= & Initial timestep. \\\hline
\verb=atol= & Absolute tolerance for step control. The default is $10^{-5}$.\\\hline
\verb=rtol= & Relative tolerance for step control. The default is $10^{-5}$.\\\hline
\verb=newton tol= & Tolerance for Newton's method. The default is $10^{-8}$.\\\hline
\verb=sparse= & Specifies whether to use sparsity when solving linear systems. This value can either be 0 or 1. \\\hline
\verb=jacobian= & The method of Jacobian calculation. The options for this value are \verb=Forward=, \verb=Centred=, \verb=Autodiff=, or \verb=Analytic=. \\\hline
\verb=jacobian splitting= & Specifies whether to apply Jacobian splitting to the IVP. This value can either be 0 or 1. \\\hline
\verb=max steps= & The maximum number of steps until simulation is stopped. \\\hline
\verb=min write time=& The minimum amount of elapsed simulation time before the next solution point can be written. A better approach is to use an interpolant that is associated with the numerical method. However, \pypp\ does not presently support an interpolated output. \\\hline
\verb=timing group=& A given parameter set must be run multiple times to conduct accurate timings. This is accomplished by specifying the parameter set hash multiple times in the run list. Each group of identical parameter sets should have a unique timing group (unique with respect to other sets of identical parameters) so the analysis phase can appropriately group runs. \\\hline
\end{tabular}
\end{center}

Analysis passes are specified by the function
\verb=GenerateAnalysisPasses()=, which similarly returns a list of
hashes. Each hash specifies the set of parameters to be used in
conducting the analysis pass. For example, to print reference
solutions, perform time versus accuracy comparisons, and perform steps
versus accuracy comparisons, the analysis function might look like:
\begin{lstlisting}[tabsize=4,language=Python]
def GenerateAnalysisPasses():
	passes = []
	for ivp in ivps:
		# Perform solution plots
		passes.append({ 'mode': 'Solutions',
						'title': ivp + ' Solutions',
						'filename': ivp + '-solutions',
						'xlabel': 'Time (s)',
						'ylabel': 'Solution',
						'legend': SolutionLegendName,
						'match': {'ivp': ivp,
								  'method': methods,
								  'atol': tolerances[-1][0] } })
		# Perform time versus accuracy plots
		passes.append({ 'mode': 'Accuracy',
						'title': ivp + ' CPU Time vs. Accuracy',
						'filename': ivp + '-cputime',
						'xlabel': 'Accuracy',
						'ylabel': 'CPU Time (ms)',
						'legend': AccuracyLegendName,
						'reference solution': reference_solutions[ivp],
						'match': {'ivp': ivp},
						'comparison': 'time',
						'group': ['method','solver','jacobian'] })
		# Perform steps versus accuracy plots
		passes.append({ 'mode': 'Accuracy',
						'title': ivp + ' Steps vs. Accuracy',
						'filename': ivp + '-steps',
						'xlabel': 'Accuracy',
						'ylabel': 'Steps',
						'legend': AccuracyLegendName,
						'reference solution': reference_solutions[ivp],
						'match': {'ivp': ivp},
						'comparison': 'steps',
						'group': ['method','solver','jacobian'] })
\end{lstlisting}


\section{Implementing a Problem}

The implementation of an IVP consists of defining the right-hand side
and the initial condition. All IVPs inherit from the base-class
\verb=BaseIVP=. It is effective to learn by example. There are many IVPs
provided with \pypp\ upon which to base future implementations. This
section gives a brief overview of the basics for implementing IVPs in
\pypp.

For the simple ODE $y'=-y, y(0)=10$ where the final simulation time is
5, an implementation might look like:

\begin{lstlisting}[tabsize=4,language=c++]
// TestEquation is inheriting from BaseIVP
class TestEquation : public BaseIVP {	
protected: 
	// Definition of the right-hand side.
	// The function name and parameters must match
	// this format exactly.
	void RHS(const FP t, const Vec<FP>& y, Vec<FP>& yp) {
		yp(0) = -y(0);
	}

public:
	// Definition of the constructor.
	// Once again, the parameters must match exactly, and
	// this function must always pass params to the BaseIVP
	TestEquation(Hash<ParamValue>& params) : BaseIVP(params) {
		// Set the (default) final time
		SetDefaultFP(params, "tf", 5.);
		 // Set the problem size to 1
		_initialCondition.Resize(1);
		// Set the initial condition to 10
		_initialCondition[0] = 10;
	}
	
	IVP_NAME("Nonstiff A1") // Macro to define IVP name
};
\end{lstlisting}

The implementation of an IVP class (as in the above example) should be
placed into a header file; this header file must be included in
\verb=loaders/ivploader.cpp=. Lastly, for the IVP to be usable,
\verb=IVPCASE(NewClassName)= must be inserted in \verb=AllocIVP()= (with
\verb=NewClassName= replaced by the name of the class). Note that this
line must come before the \verb=throw= statement.

The right-hand side of the IVP can be interpreted as additively split
when it is comprised of the sum of two or more contributing factors. A
specific class \verb=TwoSplittingIVP= inherits from \verb=BaseIVP= to
make implementations of 2-additive IVPs easy. In such cases, the user can
simply define:
\begin{itemize}
\item \verb=void Split1(const FP t, const Vec<FP>& y, Vec<FP>& yp) { ... }=
\item \verb=void Split2(const FP t, const Vec<FP>& y, Vec<FP>& yp) { ... }=
\end{itemize}

For many numerical methods, the Jacobian is required. Jacobian
matrices generated by forward or centred differences do not require
any additional work to implement; ones generated by automatic
differentiation or manually (e.g., analytical) do. The class
\verb=BaseIVP= contains a virtual function \verb=JacAnalytic= (or
\verb=JacAnalyticSparse= when using sparsity) that can be overloaded
to provide the analytic Jacobian. See the example contained in
\verb=ivps/zbinden/advection1d.h= for an IVP that is 2-additive,
supports automatic differentiation, and defines a manual analytic
Jacobian.

\section{Implementing a Method}

The implementation of a method defines how to take a step from one state to another. The following examples show how to create a single RK method and an IMEX method. The \verb=Runge2= is a second-order, explicit RK method, which is implemented as
\begin{lstlisting}[tabsize=4,language=c++]
// Inherit from the class of ERK methods
class Runge2 : public ERK {
public:
	// Constructor that specifies a Butcher tableau
	// of size 2
	Runge2(Hash<ParamValue>& params, BaseIVP* ivp)
		: ERK(params, ivp, 2) {
		_a(1,0) = 1./2;
		_b(1) = 1.;
		// Fill up the C values to make the method
		// consistent
		FillC();
	}

	// The name of method
	const char* GetName() const {
		return "Runge 2";
	}

	// Specify the order for step control	
	long GetOrder() const {
		return 2;
	}
};
\end{lstlisting}

The IMEX method specifies two Butcher tableaux, where \verb=_a= and \verb=_b= refer to the implicit tableau, and \verb=_a2= and \verb=_b2= refer to the explicit tableau. Examples are given for any method in the folder \verb=methods/ark=.

\bibliographystyle{plain}
\bibliography{tutorial}

\end{document}

